\documentclass[11pt,letter]{article}
\usepackage[top=1.00in, bottom=1.0in, left=1.1in, right=1.1in]{geometry}
\renewcommand{\baselinestretch}{1.1}
\usepackage{graphicx}
\usepackage{natbib}
\usepackage{amsmath}

\def\labelitemi{--}
\parindent=0pt

\begin{document}
\bibliographystyle{..//..//refs/besjournals}
\renewcommand{\refname}{\CHead{}}

11 September 2020\\

Singer \& Parmesan provide details of two insect-plant case study systems in their comment alongside clarifying details on how the definitions we use in our paper apply to these two systems. Although they define some terms differently and use some terms more broadly than we do, we agree with most of their points. Our paper's aim was to provide a critique of the current support for the Cushing match-mismatch hypothesis in the trophic mismatch literature and show how rarely pre-climate change baselines have been defined; thus it may not apply to systems where the Cushing hypothesis is inappropriate or when authors use alternative definitions than we do. Because their comments do not appear to conflict with our paper we do not believe any corrections or adjustments to our paper are necessary.  We provide more details below.\\

Singer \& Parmesan argue that the definition of asynchrony that we used, and which they agree is well-established, is insufficient for ``insects tasked with fitting their life cycles into time windows when hosts are edible.''  We generally agree with this idea. Indeed much of our paper is spent laying out the complexities of the Cushing hypothesis and why it may often not apply, though we argue in the paper that it may not apply for many systems---not just insect-plant ones (see Figure 2 for an overview, and much of the text). Definitions in this field are still in-flux and while Singer \& Parmesan argue for a new definition, and \citet{vissergienapp2019} have argued recently for new terms, in keeping with our goal of reviewing the current support for the Cushing hypothesis in the literature, we stand behind our definitions which are being used in the literature. Our manuscript clearly lays out the literature from which we draw our definitions, which we chose for simplicity, clarity, and because of their wide current use (e.g., we state in the paper, ``we follow Cushing's definition of mismatch, and its connections with fitness, which is in line with other studies \citep{Johansson2015,durant2007,kerby2012}''). \\

Singer \& Parmesan provide focused details on two insect plant case studies: one for winter moth on oak leaves and one for the Bay Checkerspot butterfly. For the winter moth on oak leaves system, they suggest we should have used data from Figure 5, rather than Figure 3 in \citet{tikka2003}. We agree with Singer \& Parmesan that the response variable in Figure 5 represents a more overall measure of fitness than the mortality data that we used. However, since both figures demonstrate a similar relationship between performance and synchrony, we chose Figure 3 which included their raw data. As with any case study, we are limited by the data authors present. Our figure, using the data shown in the paper, is in line with Singer \& Parmesan's point that there is a fitness cost both to being too early and too late (Figure panel a in Box 2). If their argument is more on the degree of a `fitness penalty' in the early part of the season (the shape of the curve), we agree that this appears to be the case and note that this is still broadly in line with the conclusions we make about the paper. We further agree with Singer \& Parmesan that the winter moth case study represents an excellent example of where both experimental and observational work has been done; this is why we used it as a case study in Box 2. \\

% Singer \& Parmesan suggest process-based models may not be useful given the potential for evolution to shape consumer-resource timing. We completely agree and discussed evolution throughout our paper for that reason. We did not specify that process-based models are solely ecological and thus cannot include evolution. Certainly, given recent work showing the degree of evolution that happens over what we may formerly have called `ecological' timescales \citep[e.g.,][]{lusten2018,charm2014fe,carroll2007}, including evolution in such models may be critical. \\

We were excited to see the data presented by Singer \& Parmesan in their Figure 2, which shows population data from a population of winter moth (eggs) collected before significant anthropogenic warming. These data form half of the critical baseline data we need for more systems. Combined with phenological data they would provide an important baseline dataset to better understand this system. While we understand Singer \& Parmesan's argument that changes in density must be driven by asynchrony, we believe a strong baseline dataset requires data on both fitness and phenology. As our paper reviews, we need more work that combines baselines estimates, with robust data on timing and fitness---as well as understanding of the diverse controllers on fitness in many systems. We agree that the winter moth on oak is an example of such a complex system, one which offers the opportunity to bring together datasets to provide a compelling and complete picture. The data shown here provide one more piece in the complex puzzle. \\

Finally, we apologize for any confusion of our reference to the asynchrony baseline hypothesis, which we attributed to Singer \& Parmesan. We referred to it more generally as a `hypothesis' (that we suggest more people should consider, across diverse systems) and did not describe their study system in our paper. Certainly we agree with their description of their study system, which is another interesting example where some of the complexities of eco-evolutionary dynamics in regards to synchrony have been studied. Our brief definition does not appear to be incorrect, but is also not exact to their system (and thus not exact to the timescales or other factors required for it to fit to their system).  As our paper is a cross-system review, we worked for definitions and descriptions that could be broadly applicable, thus as our glossary definition and main text states, an asynchrony baseline ``could occur for various reasons (for example, co-evolutionary arms race, other transient dynamics),'' which covers diverse reasons, including those for Singer \& Parmesan's specific system.\\

In summary, we agree with most of the points raised with Singer \& Parmesan. Their points do not conflict with our paper or require corrections to our paper, but provide details of two interesting systems, both of which we included in our paper, alongside many other systems and examples. We are excited that our paper has highlighted ways to bring together old and new data to better understand these systems. We believe they are some of the most promising systems to develop the process-based models and other approaches that we discuss in our paper. \\

By H.M. Kharouba \& E.M. Wolkovich

\begin{footnotesize}
\bibliography{..//..//refs/syncbib}
\end{footnotesize}
\end{document}