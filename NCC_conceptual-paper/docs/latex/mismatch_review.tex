\documentclass[11pt, oneside]{article}   	% use "amsart" instead of "article" for AMSLaTeX format
\usepackage{geometry}                		% See geometry.pdf to learn the layout options. There are lots.
\usepackage[round]{natbig}
\geometry{letterpaper}                   		% ... or a4paper or a5paper or ... 
%\geometry{landscape}                		% Activate for rotated page geometry
%\usepackage[parfill]{parskip}    		% Activate to begin paragraphs with an empty line rather than an indent
\usepackage{graphicx}				% Use pdf, png, jpg, or eps§ with pdflatex; use eps in DVI mode
								% TeX will automatically convert eps --> pdf in pdflatex		
\usepackage{amssymb}
%\usepackage[style=reading, abstract-false, abbreviate=false, backend=bibtex]{biblatex} % imports package biblatex (a biblio management package)
%\addbibresource{sample.bib} %Imports the bibtex data file sample.bib, this file is the one that includes information about each referenced book, article, etc.

%SetFonts

%SetFonts


\title{Trophic phenological mismatch: Disconnects between underlying ecological theory and climate change responses}
\author{Heather M. Kharouba1, Elizabeth M. Wolkovich2,3}
\date{}						% Activate to display a given date or no date

\begin{document}
%\begin[flushleft]
\maketitle
\section{Abstract}
\setlength{\parindent}{10ex}
\indent
Many researchers hypothesize that climate change will lead to phenological mismatches with negative consequences for those interacting species and their ecological communities; yet, evidence documenting negative impacts on fitness is mixed. The most common ecological theory that underlies these studies is the Cushing match-mismatch hypothesis. It offers a testable hypothesis for predicting these consequences due to climate change. Here, we conduct a literature review and find that X of studies fail to collect data to provide strong tests of this hypothesis, thus making it difficult to assess support for this major hypothesis. Further, we find that X of studies fail to define pre-climate change baselines in their study system, making predictions difficult. To accurately predict the magnitude and prevalence of mismatches due to climate change, relating empirical observations to underlying mechanisms through hypothesis testing will be required. By adjusting their study designs, researchers can more rigorously test this hypothesis. We highlight how these approaches could rapidly advance our mechanistic understanding and thus allow robust predictions of shifts with continuing climate change. 

\section{Introduction}

Climate change is causing phenological shifts (i.e. changes in the timing of life history events) that vary across species in different functional groups and trophic levels (\citep{thackeray2016}; Ovaskainen et al. 2013; CaraDonna et al. 2014). Such species-specific variation in response to climate change has led to changes in the relative timing of key activities (phenological synchrony) among interacting species (Kharouba et al. 2018). These changes have caused fitness consequences—often termed ‘phenological mismatch’ (Box 1)—and have influenced ecosystem-level properties in some contexts (Post and Forchhammer 2008; Plard et al. 2014; Doiron et al. 2015; Burkle et al. 2013) but not others (Vatka et al. 2011; Burthe et al. 2012). Despite many theoretical (Bewick et al. 2016; Johansson et al. 2015) and empirical studies (REF) based in single systems, we still have no general ability to predict the outcomes of shifts in phenological synchrony due to climate change.\par

Here, we argue that much of the difficulty in predicting the consequences of climate change-driven shifts in synchrony is due to a disconnect between ecological theory and current empirical approaches in the phenological mismatch literature. Current methodological inconsistencies across studies make it difficult to test the relevant underlying ecological theory in the context of climate change. Without better evidence, we cannot attribute variation in findings across studies to species, site, or mechanism. Without an understanding of the mechanisms underlying the well-documented patterns in phenological shifts, our ability to make accurate predictions about species’ responses, and species’ interactions, to climate change remains limited (O’Connor et al. 2012; Chmura et al. 2018). \par

Here, we focus on the widely-cited Cushing match-mismatch, or trophic mismatch, hypothesis (1974), the most commonly applied hypothesis to consumer-resource interactions in this literature. We show how advances could come from direct tests of the hypothesis and clear definitions of baselines, when possible. Our aim is not to put forward additional hypotheses about the context in which phenological mismatch will occur, which has been reviewed extensively elsewhere (e.g., Miller-Rushing 2010; Renner and Zohner 2018), but rather to help guide the study of phenological mismatch to develop more robust predictions. \par

Although the Cushing hypothesis has been applied to other types of interactions (e.g. mutualism), we limit our discussion to consumer-resource interactions (i.e. antagonistic). Below, we provide an overview of the Cushing hypothesis, summarize our literature review of phenological mismatch and then outline the divide between the hypothesis and the empirical studies. We discuss how current approaches are impeding major progress in the field but that changes in our approach could rapidly advance our understanding and help forecast of the impacts of climate change on ecological communities, the ultimate goal of most of the phenological mismatch literature. 

\subsection{Overview of the main ecological theory}
The most common ecological theory that underlies phenological mismatch studies (Appendix) is the Cushing match-mismatch hypothesis. This hypothesis predicts the often-shown concave down curve between consumer fitness and relative timing between the consumer and its resource (1974; Figure 1). While this curve has been applied across many ecosystems (CITES), the theory originally emerged from the marine fisheries literature as a way to explain the variation in population recruitment of fish stocks. 

\section{Disconnect between theory and empirical studies}
In its original state, the hypothesis has been debated, contested and criticized, particularly in the marine literature (Durant et al. 2007, Leggett and DeBlois 1994*). In part because, although a relatively simple hypothesis, it is inherently difficult to test in the field, an assertion even Cushing himself made. Indeed, the shape and strength of the relationship of the curve varies greatly across observational studies (e.g., Philippart et al. 2013; Reed et al. 2013; Plard et al. 2014; Atkinson et al. 2015). While others have suggested that this is because of data limitations and the model’s implication of complex multitrophic dynamics (Kerby chapter, Durant et al. 2007), we argue that there are key methodological reasons that make it difficult to determine whether this hypothesis is widely supported in the context of climate change. Below, we introduce the current objectives of the phenological mismatch literature, and then discuss how studies often fail to rigorously test the Cushing hypothesis. We also examine whether studies define pre-climate change baselines, which are critical for assessing climate change impacts now, and in the future.

\subsection{Testing fundamental theory}

The Cushing hypothesis offers a testable, generally applicable hypothesis for predicting the magnitude and direction of demographic changes in response to climate-change driven shifts in synchrony (Figure 2). To date, much research in the biological impacts of climate change literature has focused on the direct relationships between organisms and the environment (e.g., Menzel et al. 2006, Chen et al. 2011) rather than testing theory (Lavergne et al. 2010; O’Connor et al. 2012; Mouquet et al. 2015; Barner et al. 2018). However, progress on the Cushing hypothesis requires tests of a diversity of ecological and evolutionary theory. This represents the major challenge of the hypothesis and—we argue—may be why support for it has been so mixed. 

\bibliographystyle{plainnatt}
\bibliography{mismatch}
%\end{flushleft}
\end{document}  